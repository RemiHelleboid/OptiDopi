\documentclass[]{article}

\usepackage{amsmath}
\usepackage{amssymb}

% r vector
\newcommand{\rv}{\mathbf{r}}

\title{Poisson}

\begin{document}

\maketitle

\section{Poisson}

\subsection{Main equations}

The Poisson equation for semiconductors simulation is given by:
\begin{equation}
\label{eq:poisson}
\nabla \cdot \left( \epsilon \nabla \phi \right) = q \left( N_D - N_A + N_C^+ - N_V^- \right)
\end{equation}
where $\epsilon$ is the dielectric constant, $\phi$ is the electrostatic potential, $q$ is the elementary charge, $N_D$ is the donor concentration, $N_A$ is the acceptor concentration, $N_C^+$ is the concentration of ionized donors and $N_V^-$ is the concentration of ionized acceptors.

For now, we will focus on the case of a bulk and unique semiconductor material, so that $\epsilon$ is constant over the whole domain. 
We shall note $\rho = q \left( N_D - N_A + N_C^+ - N_V^- \right)$ the charge density, so that we are left with:

\begin{equation}
\label{eq:poisson2}
\Delta \phi(\rv) = \frac{\rho(\rv)}{\epsilon}
\end{equation}

We shall consider two types of boundary conditions:
\begin{itemize}
\item Dirichlet boundary conditions: $\phi(\rv) = \phi_D(\rv)$ on $\partial \Omega_D$
\item Neumann boundary conditions: $\frac{\partial \phi}{\partial n} = \phi_N(\rv)$ on $\partial \Omega_N$
\end{itemize}
where $\partial \Omega_D$ and $\partial \Omega_N$ are the Dirichlet and Neumann boundaries respectively, and $n$ is the normal vector to the boundary.

At the contacts, we consider that the potential is fixed, so that we have Dirichlet boundary conditions:
\begin{equation}
\label{eq:poisson3}
\phi(\rv) = \phi_D(\rv) \quad \text{on} \quad \partial \Omega_D.
\end{equation}

At the interfaces with insulators (air, oxide), we consider that the component of the electric field normal to the interface is null, so that we have Neumann boundary conditions:
\begin{equation}
\label{eq:poisson4}
\frac{\partial \phi}{\partial n} = 0 \quad \text{on} \quad \partial \Omega_N.
\end{equation}

\subsection{Charge density}
We place ourselves in the case of a non-degenerate semiconductor, at equilibrium, so that the charge density for electrons and holes is given by:
\begin{equation}
\label{eq:charge_density}
n = n_i \exp \left( \frac{\phi}{V_T} \right) \quad \text{and} \quad p = n_i \exp \left( - \frac{\phi}{V_T} \right)
\end{equation}
where $n_i$ is the intrinsic carrier concentration, $\phi$ is the electrostatic potential, and $V_T$ is the thermal voltage.

In the presence of an applied voltage, the charge density is given by:
\begin{equation}
\label{eq:charge_density2}
n = n_i \exp \left( \frac{\phi - V}{V_T} \right) \quad \text{and} \quad p = n_i \exp \left( \frac{V - \phi}{V_T} \right)
\end{equation}
where $V$ is the applied voltage.


\end{document}